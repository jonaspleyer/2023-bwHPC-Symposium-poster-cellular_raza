%%%%%%%%%%%%%%%%%%%%%%%%%%%%%%%
% Template for bwHPC-Symposium 
% (compile with pdflatex)
%%%%%%%%%%%%%%%%%%%%%%%%%%%%%%%

%
% --- do NOT change anything here ---
%

\documentclass[fontsize=11pt,a4paper]{article}
\usepackage[T1]{fontenc}
\usepackage[utf8]{inputenc}
\usepackage[english]{babel}
\usepackage{graphicx}
\usepackage[a4paper]{geometry}
\geometry{outer=2.5cm, inner=2.5cm, top=3.5cm, bottom=3.5cm}
\usepackage{lmodern}
\usepackage{lineno}
\usepackage{color}
\newcommand{\red}[1]{\textcolor{red}{#1}}
\newcommand{\changed}[1]{{#1}}
\newcommand*{\REVIEW}{} % comment this out in the final version of the document
\ifdefined\REVIEW
   \linenumbers
   \renewcommand{\changed}[1]{\textcolor{red}{#1}}
\fi

% switch font family to using sans-serif by default for everything (except mathematics)
\renewcommand{\familydefault}{\sfdefault}

%
% --- optional packages ---
%

%\usepackage{verbatim}
%\usepackage{hyperref}
%\usepackage{amsmath} 
%\usepackage{amsfonts} 
%\usepackage{amssymb}

%
% --- title and author ---
%

\title{Title of the extended abstract}

\author{
        Main Author \\
        Institute A \\
        University X
            \and
        Co Author \\
        Institute B \\
        University Y
}

%
% --- print date and title ---
%

\date{}
\begin{document}
\maketitle

%
% --- main contents starts here ---
%

\begin{abstract}
A typical abstract is 5-10 lines long. The entire document should contain 3 to 6 pages.
\end{abstract}

\section{Introduction}\label{introduction}

The following sample sections are only suggestions –- please feel free to name
them as you wish, add/remove (sub-)sections, etc.

\section{Theory and methods}\label{theory}

Theoretical, methodical and/or experimental details. Example
reference \cite{muster} -- for the list of references see end of document.

\section{Results and discussion}\label{results}

The main focus of this section is the presentation of the scientific results
including diagrams, figures, tables, etc.

Additionally some HPC related details are very much appreciated, for example
information about CPU usage, number of cores per job, scaling behavior,
parallelization efficiency, typical memory usage, temporary disk requirements,
or similar information interesting for HPC focused readers.

%
% --- figure example ---
%

Figures owned by someone else may only be used with permission and must be
referenced properly in the figure caption. Do not let text flow around figures.

% \begin{figure}[htb] 
%   \begin{center}
%     \includegraphics[scale=0.5]{bwHPC_Logo_rgb.jpeg}
%     \caption{Our logo}
%     \label{fig:bwlogo}
%   \end{center}
% \end{figure}

\section{Conclusions}\label{conclusions}

Conclude the findings. 

\section{Acknowledgements} 

Resources that did contribute to this research.

%\bibliographystyle{abbrv}
%\bibliography{simple}

\begin{thebibliography}{}

\bibitem{muster}
M. Mustermann, N. Nobody, and J. Doe. The joy of publishing. Journal of Any Pub. 32 (2014) 575--582.

\end{thebibliography}

\end{document} 


%%% Local Variables:
%%% mode: latex
%%% TeX-master: t
%%% End:

